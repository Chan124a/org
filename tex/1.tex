\documentclass[UTF8]{ctexart}
\usepackage{geometry}
\geometry{a4paper,left=1cm,right=1cm,top=1cm,bottom=1cm}
\usepackage{mathtools}
\begin{document}
设输入空间为\textsl{n}维向量的集合,输出空间为类标记集合${c_1,c_2,\cdots,c_k}$.输入为特征向量\textsl{x}属于输入空间,输出为类标记\textsl{y}属于输出空间.\textsl{X}s是定义在输入空间上的随机变量,\textsl{Y}是定义在在输出空间上的随机向量.\textsl{P(X,Y)}是\textsl{X}和\textsl{Y}的联合概率分布.

训练数据集为:$T={(x_1,y_{1}),(x_2,y_2),\cdots,(x_N,y_N)}$
  由\textsl{P(X,Y)}独立同分布产生.

  朴素贝叶斯通过训练数据集学习联合概率分布 \textsl{P(X,Y)}.
  具体地,学习以下先验概率分布及条件概率分布:先验概率分布$P(Y=c_k),k=1,2,\cdots,K$,条件概率分布$P(X=x|Y=c_k)=P(X^{(1)}=x^{(1)},\cdots,X^{(n)}=x^{(n)}|Y=c_{k}),k=1,2,\cdots,K$

  于是学习到联合概率分布\textsl{P(X,Y)}

  假设$x^{(j)}$可取值有$S_j$个,则$x^{(j)}$可能取值的集合为${a_{j1},a_{j2},\cdots,a_{jS_{j}}}$, \textsl{j=1,2,...n};\textsl{Y} 可取值有\textsl{K} 个,那么参数个数为$K \prod^n_{j=1} S_j$

  朴素贝叶斯法对条件概率分布作了条件独立性的假设.

  $P(X=x|Y=c_k)=P(X^{(1)}=x^{(1)},\cdots,X^{(n)}=x^{(n)}|Y=c_{k})= \prod^n_{j=1} P(X^{(j)}=x^{(j)}|Y=c_{k})$

  朴素贝叶斯法分类时,对给定的输入\textsl{x},通过学习到的模型计算后验概率分布$P(Y=c_k|X=x)$,将后验概率最大的类作为输出.后验概率计算根据贝叶斯定理进行:

$P(Y=c_k|X=x)= \frac{P(Y=c_k) \prod^n_{j=1} P(X^{(j)}=x^{(j)}|Y=c_{k})}{ \sum^K_{k=1}  P(X=x|Y=c_{k}) P(Y=c_k) }$

将条件独立性假设代入上式得:

$P(Y=c_k|X=x)= \frac{P(Y=c_k) \prod^n_{j=1} P(X^{(j)}=x^{(j)}|Y=c_{k})}{\sum^K_{k=1} P(Y=c_k) \prod^n_{j=1} P(X^{(j)}=x^{(j)}|Y=c_{k})}$

这是朴素贝叶斯分类的基本公式.于是,朴素贝叶斯分类器可表示为:

$y=f(x)=\mathop{\arg\max}\limits_{c_{k}} \frac{P(Y=c_k) \prod^n_{j=1} P(X^{(j)}=x^{(j)}|Y=c_{k})}{\sum^K_{k=1} P(Y=c_k) \prod^n_{j=1} P(X^{(j)}=x^{(j)}|Y=c_{k})}$

注意到分母对所有$c_k$都是相同的,所以

$y=f(x)=\mathop{\arg\max}\limits_{c_{k}} P(Y=c_k) \prod^n_{j=1} P(X^{(j)}=x^{(j)}|Y=c_{k})$

从上式可以看出,朴素贝叶斯法的学习也就是要估计先验概率$P(Y=c_k)$和条件分布概率$P(X^{(j)}=x^{(j)}|Y=c_{k})$,可以应用极大似然估计法估计相应的概率.

下面写出推导过程:

把$p(Y=c_k)$和$p(x^{(j)}=a_{jl}|y=c_{k})$作为参数.

$p(y)=\prod^K_{k=1} p(y=c_k)^{1 \{y=c_{k}\} }$

$p(x|y=c_k)= \prod^n_{j=1}p(x^j|y=c_k)= \prod^n_{j=1} \prod^{S_j}_{l=1} p(X^{(j)}=a_{jl}|Y=c_{k})^{1 \{x^{(j)}=a_{jl},y=c_k\} }$

为叙述方便起见,下面以$\varphi$代表参数集合${p(Y=c_k),p(x^{(j)}=a_{jl}|y=c_{k})}$

先写出log似然函数

\begin{equation*}
  \begin{split}
l(\varphi) &=log \prod^N_{i=1} p(x_i,y_i;\varphi) \\
             &=log \prod^N_{i=1} p(x_i;y_i;\varphi)p(y_i;\varphi) \\
            &=log \prod^N_{i=1}[\prod^n_{j=1} p(x^{(j)}_i | y_i;\varphi) ]p(y_i;\varphi) \\
            &= \sum^N_{i=1}(log p(y_i,\varphi)+ \sum^n_{j=1} log p(x^{(j)}_i|y_i;\varphi)) \\
            &=\sum^N_{i=1}[\sum^K_{k=1} log p(y=c_k)^{1\{y_i=c_k\} }+ \sum^{K}_{k=1}\sum^n_{j=1} \sum^{S_j}_{l=1} log p(x^j=a_{jl}|y=c_k)^{1\{x^{(j)}_i=a_{jl},y_i=c_k\} }] \\
            &=\sum^N_{i=1}[\sum^K_{k=1} 1\{y_i=c_k\} log p(y=c_k)+\sum^{K}_{k=1} \sum^n_{j=1} \sum^{S_j}_{l=1} 1\{x^{(j)}_i=a_{jl},y_i=c_k\} logp(x^j=a_{jl}|y=c_k)]
        \end{split}
\end{equation*}

在上式中把$p(Y=c_k)\mbox{和} p(x^{(j)}=a_{jl}|y=c_{k})(j=1,2,\cdots,n;l=1,2,\cdots,S_j;k=1,2,\cdots,K)$作为参数.

先求先验概率$p(Y=c_k)$的最大似然估计,因为存在约束条件$\sum^K_{k=1} p(y=c_k)=1$,所以下面开始用拉格朗日乘数法分别求最大似然估计(条件极值):

上式中只有前半段含有$p(Y=c_k)$,所以在求先验概率估计值时就只管前半部分.

令$F=\sum\limits^N_{i=1} \Bigl\{ \bigl(\sum\limits^K_{k=1} 1\{y_i=c_k\}log p(y=c_k)\bigr)+ \lambda (1-\sum\limits^K_{k=1} p(y=c_k) ) \Bigr\}$

分别对$p(y=c_{k})(k=1,2,\cdots,K) \mbox{和} \lambda$求导:

\begin{equation}
  \left\{ \begin{lgathered}
      \frac{\partial F}{\partial p(y=c_{1})} = \sum\limits^N_{i=1} \{\frac{1\{y_i=c_{1}\}}{p(y=c_{1})} -\lambda\}=0 \\
      \frac{\partial F}{\partial p(y=c_{2})}= \sum\limits^N_{i=1} \{\frac{1\{y_i=c_{2}\}}{p(y=c_{2})} -\lambda\}=0 \\
              \vdots \\
              \frac{\partial F}{\partial p(y=c_{K})}= \sum\limits^N_{i=1} \{\frac{1\{y_i=c_{K}\}}{p(y=c_{K})} -\lambda\}=0 \\
      \frac{\partial F}{\partial \lambda } =\sum\limits^N_{i=1}\{1-\sum\limits^K_{k=1} p(y=c_k)        \}=0
      \end{lgathered} \right.
\end{equation}

则由前面面$K$个式子可得:
\begin{equation}
  \begin{cases}
    p(y=c_{1})=\frac{\sum\limits^N_{i=1} 1\{y_i=c_{1}\}}{N\lambda} \\
    p(y=c_{2})=\frac{\sum\limits^N_{i=1} 1\{y_i=c_{2}\}}{N\lambda}\\
    \vdots\\
    p(y=c_{K})=\frac{\sum\limits^N_{i=1} 1\{y_i=c_{K}\}}{N\lambda}
    \end{cases}
\end{equation}

由于$\sum\limits^K_{k=1} p(y=c_k)=1$,则将上面左边全部式子加起来,可以得到
  $$1=\sum\limits^K_{k=1} p(y=c_k)=\frac{\sum\limits^{K}_{k=1} \sum\limits^{N}_{i=1} 1\{y_{i}=c_{k}\}}{N\lambda} =\frac{N}{N\lambda} $$

即$\lambda =1$,代入方程组(2),可得$p(y=c_k)$的极大似然估计为:

$p(y=c_k)= \frac{\sum\limits^N_{i=1} 1\{y_i=c_{2}\}}{N}(k=1,2,\cdots,K)$

下面开始求$p(x^{(j)}=a_{jl}|y=c_{k})$的极大似然估计:

已知log似然函数为:

$$
l(\varphi) =\sum^N_{i=1}[\sum^K_{k=1} 1\{y_i=c_k\}log p(y=c_k)+\sum^K_{k=1}\sum^n_{j=1} \sum^{S_j}_{l=1} 1\{x^{(j)}_i=a_{jl},y_i=c_k\} logp(x^{(j)}=a_{jl}|y=c_k)] $$

只需对式子后面部分求偏导即可.由于存在约束条件$\sum\limits^{S_j}_{l=1}p(x^{(j)}=a_{jl}|y=c_k)=1$,所以也可用拉格朗日乘数法求极大似然估计:

令$$G=\sum^N_{i=1} \biggl\{ \sum^K_{k=1} \sum^n_{j=1} \Bigl(  \bigl( \sum^{S_j}_{l=1} 1\{x^{(j)}_i=a_{jl},y_i=c_k\} logp(x^{(j)}=a_{jl}|y=c_k) \bigr) + \lambda_{kj} (1- \sum\limits^{S_j}_{l=1} p(x^{(j)}=a_{jl}|y=c_k)) \Bigr) \biggr\}$$

注意由于对于每个\textsl{k}和\textsl{j}都存在约束条件$\sum\limits^{S_j}_{l=1}p(x^{(j)}=a_{jl}|y=c_k)=1$,所以总共有 $k\times l $ 个约束条件,上式中的参数$\lambda_{kj}$对应的是\textsl{k}和\textsl{j}固定时的约束条件.

\begin{equation}
  \left\{ \begin{lgathered}
      \frac{\partial G}{\partial p(x^{(j)}=a_{jl}|y=c_k)} = \sum\limits^N_{i=1} \Bigl\{  \frac{ 1 \{ x^{(j)}_i=a_{jl},y_i=c_k \} }{ p( x^j = a_{jl} | y=c_k ) } - \lambda_{kj} \Bigr\}=0 \\
      \frac{\partial G}{\partial \lambda_{kj} }= \sum\limits^N_{i=1}  \{ 1 - \sum\limits^{S_j}_l p(x^{(j)} =a_{jl} | y=c_k ) \}=0 
      \end{lgathered} \right.
  \end{equation}

  由第1个式子可得$ p( x^j = a_{jl} | y=c_k ) = \frac{ \sum\limits^N_{i=1} 1 \{ x^{(j)}_i=a_{jl},y_i=c_k \} }{ N \lambda_{kj} }$

  由第2个式子可得$ \sum\limits^{S_j}_l p(x^{(j)} =a_{jl} | y=c_k ) = 1 $

  联立两个式子可以得到:

  $1= \sum\limits^{S_j}_l p(x^{(j)} =a_{jl} | y=c_k ) =  \frac{  \sum\limits^{S_j}_l \sum\limits^N_{i=1} 1 \{ x^{(j)}_i=a_{jl},y_i=c_k \} }{ N \lambda_{kj} } = \frac{ \sum\limits^N_{i=1} 1 \{ y_i=c_k \} }{ N \lambda_{kj} }$

  解得:$ N \lambda_{kj} = \sum\limits^N_{i=1} 1 \{ y_i=c_k \} $

  则有:$ p( x^j = a_{jl} | y=c_k ) = \frac{  \sum\limits^N_{i=1} 1 \{ x^{(j)}_i=a_{jl},y_i=c_k \} }{ \sum\limits^N_{i=1} 1 \{ y_i=c_k \} }$

  证明完毕.
\end{document}
%%% Local Variables:
%%% mode: latex
%%% TeX-master: t
%%% End:
