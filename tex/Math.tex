\documentclass[UTF8]{ctexart}
\newtheorem{theorem}{定理}[section]
\newtheorem{pro}{命题}[section]
\newtheorem{coro}{推论}[theorem]
\newtheorem{lemma}{引理}[section]
\begin{document}
\section{定义}
\begin{enumerate}
\item R中稠密 P4
\item 数列收敛 P9,13
\item 上界,下界,有界数列 P13
\item 子列 P14
\item 数列极限正负无穷大的定义 P24
\item 单调数列 P26
\item 基本类 P36
\item 上下确界 P40
\item 开覆盖 P43
\item 上下极限 P45
\item 映射,定义域,像,原像 P55
\item 相等映射,逆映射 P56
\item 满射,单射,一一对应 P57
\item 映射的复合 P57
\item 恒等映射 P58
\item 集合的等价 P59
\item 有限集,无限集,可数集,不可数集,至多可数集 P59
\item 函数,反函数 P63
\item (严格)单调函数,(严格)递增函数,(严格)递减函数 P66
\item 函数$f$在点$x_{0}$有极限$l$ P69
\item 单边极限,左极限,右极限 P75
\item 在无穷处的函数极限 P81
\item 无穷小,无穷大 P84
\item 高阶无穷小,同阶无穷小,等价无穷小 P85
\item 高阶无穷大,同阶无穷大,等价无穷大 P85
\item $O(g(x)) \mbox{与} o(g(x))$的定义. P88
\item 函数在点$x_0$处连续 P90
\item 左连续,右连续 P92
\item $f$在$I$上连续 P93
\end{enumerate}

\section{实数}
任何分数一定是有尽小数或无尽循环小数。
	
每一个实数都可以用有理数去逼近到任意精确的程度。
	
有理数集Q在R中是稠密的。
	
\section{数列极限}
\begin{theorem}
    	如果数列$\{a_n\}$收敛,则它只有一个极限。
\end{theorem}
\begin{theorem}
收敛数列是有界的。
\end{theorem}
\begin{theorem}
设收敛数列$\{a_n\}$的极限是$a$,那么$\{a_n\}$的任何一个子列都收敛于$a$。
\end{theorem}
\begin{coro}
		数列$\{a_n\}$收敛的充分必要条件是它的偶数项子列$\{a_{2n}\}$和奇数项子列$\{a_{2n-1}\}$都收敛,并且有相同的极限。
\end{coro}
\begin{theorem}(极限的四则运算)
	设$\{a_n\}$与$\{b_n\}$都是收敛数列,则${a_n+b_n}$,${a_nb_b}$也是收敛数列。如果$\lim\limits_{n\to\infty}b_n\neq0$,则$\{a_n/b_n\}$也收敛,并且有:
\begin{enumerate}
	\item $\lim\limits_{n\to\infty}a_n+b_n=\lim\limits_{n\to\infty}a_n$;
	\item $\lim\limits_{n\to\infty}a_nb_n=\lim\limits_{n\to\infty}a_n \cdot \lim\limits_{n\to\infty}b_n$,特别的,如果$c$是常数,便有$\lim\limits_{n\to\infty}ca_n=c\lim\limits_{n\to\infty}a_n;$
	\item $\lim\limits_{n\to\infty} \frac{a_n}{b_n}=\frac{\lim\limits_{n\to\infty}a_n}{\lim\limits_{n\to\infty}b_n}$,其中$\lim\limits_{n\to\infty}b_n\neq0.$	
\end{enumerate}
\end{theorem}
\begin{theorem}
(夹逼定理)设$a_n\leq b_n \leq c_n (n\in N_*)$,如果$\lim\limits_{n\to\infty}a_n=\lim\limits_{n\to\infty}c_n=a$,那么$\lim\limits_{n\to\infty}b_n=a$
\end{theorem}
\begin{theorem}保号性
\begin{enumerate}
	\item 设$\lim\limits_{n\to\infty}a_n=a,\alpha,\beta$满足$\alpha<a<\beta$,那么当$n$充分大时,有$a_n>\alpha$;同样,当$n$充分大时,有$a_n<\beta$
	\item 设$\lim\limits_{n\to\infty}a_n=a,\lim\limits_{n\to\infty}b_n=b$,且$a<b$,那么当$n$充分大时,一定有$a_n<b_n$.
	\item 设$\lim\limits_{n\to\infty}a_n=a,\lim\limits_{n\to\infty}b_n=b$,并且当$n$充分大时$a_n\leq b_n$,那么有$a \leq b$.
\end{enumerate}
\end{theorem}
\begin{pro}无穷大的性质
	\begin{enumerate}
		\item 如果$\{a_n\}$是无穷大,那么$\{a_n\}$必然无界.
		\item 从无界数列中一定能选出一个子列是无穷大.
		\item 如果$\lim\limits_{n\to\infty}a_n=+\infty$(或$-\infty,\infty$),那么对$\{a_n\}$的任意子列$\{a_{k_n}\}$,也有\[\lim\limits_{n\to\infty}a_{k_n}=+\infty\mbox{(或}-\infty,\infty)\].
		\item $\mbox{如果}\lim\limits_{n\to\infty}a_n=+\infty,\lim\limits_{n\to\infty}b_n=+\infty,\mbox{那么}$\[\lim\limits_{n\to\infty}(a_n+b_n)=+\infty,\lim\limits_{n\to\infty}a_nb_n=+\infty\]
		\item $\{a_n\}$是无穷大的充分必要条件是$\{1/a_n\}$为无穷小.
	\end{enumerate}
\end{pro}
\begin{theorem}
	单调有界数列一定有极限.
\end{theorem}
\begin{theorem}
	(闭区间套定理)设$I_n=[a_n,b_n](n\in N^*),\mbox{并且}I_1\supset I_2 \supset I_3 \supset \cdots \supset I_n\supset I_{n+1}\supset \cdots.$如果这一列区间的长度$\langle I_n\langle=b_n-a_n\to 0(n\to \infty),$那么交集$\bigcap\limits_{n=1}^\infty I_n$含有唯一的一点.
\end{theorem}
\begin{theorem}
	自然对数的底是无理数.
\end{theorem}
\begin{lemma}
	从任一数列中必可取出一个单调数列.
\end{lemma}
\begin{theorem}
	(列紧性定理)从任何有界数列必可选出一个收敛子列.
\end{theorem}
\begin{theorem}(Cauchy收敛定理)
	一个数列收敛的充分必要条件是,它是基本列.
\end{theorem}
\begin{theorem}(确界定理)
	非空有上界的集合必有上确界.
	
	非空有下界的集合必有下确界.
\end{theorem}
\begin{pro}
	$-sup(-E)=inf E\quad \mbox{或}\quad sup(-E)=-infE$ 
\end{pro}
\begin{theorem}
		(紧致性定理,有限覆盖定理,Heine-Borel定理)设$[a,b]$是一个有限闭区间,并且它有一个开覆盖$\{I_\lambda\}$,那么从这个开区间族必可选出有限个成员(开区间)来,这有限个开区间所成的族仍是$[a,b]$的开覆盖.
\end{theorem}	
\begin{theorem}
		设$\{a_n\}$为一数列,$E$为$\{a_n\}$所有极限点组成的集合,$a^*$为上极限.那么:
	\begin{enumerate}
		\item $a^*\in E;$
		\item $\mbox{若} x>a^*,\mbox{则存在}N\in N^*,\mbox{使得当}n\leq N\mbox{时,有}a_n<x; $
		\item $a^*$是满足前两条性质的唯一数.
	\end{enumerate}
	对下极限$a_*$也有类似定理.
\end{theorem}
\begin{theorem}
	设$\{a_n\},\{\b_n\}$是两个数列.
\begin{enumerate}
	\item $\liminf\limits_{n\to \infty} a_n\leq \limsup\limits_{n\to \infty} a_n$
	\item $\lim\limits_{n\to \infty}a_n=a$当且仅当$\liminf\limits_{n\to \infty}a_n=\limsup\limits_{n\to \infty}a_n=a;$
	\item 若\textup{N}是某个正整数,当$n>N$时,$a_\leq b_n$,那么\[\liminf\limits_{n\to \infty}a_n\leq\liminf\limits_{n\to \infty}b_n,\limsup\limits_{n\to \infty}a_n\leq\limsup\limits_{n\to \infty}b_n\]
\end{enumerate}
\end{theorem}
\begin{theorem}
	对数列$\{a_n\}$,定义$\alpha_n=\inf\limits_{k\geq n}a_k,\beta_n=\sup\limits_{k\geq n}a_k$,那么:
	\begin{enumerate}
		\item $\{\alpha_n\}$是递增数列,$\{\beta_n\}$是递减数列;
		\item $\lim\limits_{n\to \infty}\alpha_n=a_*,\lim\limits_{n\to \infty}\beta_n=a^*$.
	\end{enumerate}
\end{theorem}
\begin{theorem}
  (Sotlz定理)$\frac{\infty}{\infty}\mbox{型}$设${b_n}$是严格递增且趋于$+\infty$的数列.如果
  $$\lim\limits_{n \to \infty} \frac{a_n-a_{n-1}}{b_n-b_{n-1}}=A$$
  那么
  $$\lim\limits_{n \to \infty}\frac{a_n}{b_n}=A$$其中A可以是$+\infty \mbox{或} -\infty$.
\end{theorem}

\section{集合}
\begin{theorem}
  可数集A的每一个无限子集是可数集.
\end{theorem}
\begin{theorem}
  设${E_n}(n=1,2,3,\cdots)$是一列至多可数集.令
  \[S=\bigcup\limits_{n=1}^\infty E_n\]
  那么$S$是至多可数集.
\end{theorem}
\begin{theorem}
  $R$中的全体有理数是可数的.
\end{theorem}
\begin{theorem}
  $[0,1]$上的全体实数是不可数的
\end{theorem}



\section{函数的连续性}

\subsection{函数}
\begin{theorem}
  设函数$f$在其定义域$X$上是严格递增(递减)的,那么反函数$f^{-1}$必存在,$f^{-1}$的定义域为$f(X)$,并且$f^{-1}$在这一集合上也是严格递增(递减)的.
\end{theorem}

\subsection{函数的极限}
\begin{theorem}
  函数$f$在$x_{0}$处有极限$l$的充分必要条件是,对任意一个收敛于$x_{0}$的数列$\{x_n \neq x_0:n=1,2,3,\cdots\}$,数列${f(x_n)}$有极限$l$.
\end{theorem}
\begin{theorem}
  (函数极限的唯一性)若$\lim\limits_{x \to x_{0} }f(x)$存在,则它是唯一的. 
\end{theorem}
\begin{theorem}
  若$f$在$x_0$有极限,那么$f$在$x_0$的一个近旁是有界的.也就是说,存在整数$M \mbox{及}\delta$,使得当$0<|x-x_0|<\delta$时,$f(x)<M$.
\end{theorem}
\begin{theorem}
  设$\lim\limits_{x \to x_0 }f(x)\mbox{与}\lim\limits_{x \to x_0 }g(x)$存在,那么有:
  
\begin{itemize}
\item $\lim\limits_{x \to x_0 }(f+g)(x)=\lim\limits_{x \to x_0 }f(x) \pm \lim\limits_{x \to x_0 }g(x)$;
\item $\lim\limits_{x \to x_0 }fg(x)=\lim\limits_{x \to x_0 }f(x) \cdot \lim\limits_{x \to x_0 }g(x)$;
\item $\lim\limits_{x \to x_0 }\frac{f}{g}(x)=\frac{\lim\limits_{x \to x_0 }f(x)}{\lim\limits_{x \to x_0 }g(x)}$,其中$\lim\limits_{x \to x_0 }g(x) \neq 0$.
\end{itemize}
\end{theorem}
\begin{theorem}
  (夹逼定理)设函数$f,g \mbox{与}h$在点$x_0$的近旁(点$x_0$自身可能是例外)满足不等式
  \[f(x) \leq h(x)\leq g(x)\].
  如果$f \mbox{与} g$在点$x_0$有相同的极限$l$,那么函数$h$在点$x_0$也有极限$l$.
\end{theorem}
\begin{theorem}
  设存在$r >0$使得当$0<|x-x_0|<r$时,不等式$f(x)\leq g(x)$成立.有设在$x_0$处这两个函数都有极限,那么$\lim\limits_{x \to x_0 }f(x)\leq \lim\limits_{x \to x_0 }g(x)$
\end{theorem}
\begin{theorem}
 函数$f$在$x_0$处有极限,必须且只需对任意给定的$\epsilon >0$,存在$\delta>0$,使得对任意的$x_1,x_2 \in B_{\delta}(\check x_{0})$,都有$|f(x_1)-f(x_2)|< \epsilon.$
\end{theorem}
\begin{theorem}
 设$\lim\limits_{x \to x_0 }f(x)=l,\lim\limits_{t \to t_0 }g(t)=x_0$.如果在$t_0$的某个领域$B_{\eta}(t_0)$内$g(t) \neq x_0$,那么$\lim\limits_{t \to t_0 }f(g(t))=l$.
\end{theorem}
\begin{theorem}
  设函数$f$在$x_0$的某个领域内($x_0$可能是例外)有定义,那么$\lim\limits_{x \to x_0 }f(x)$存在的充分必要条件是
  \[ f(x_0+)=f(x_0-) \]
  这个共同的值也就是函数$f$在$x_0$处的极限值.
\end{theorem}
\begin{theorem}
  如果当$x\rightarrow x_0$ ($x_0$可以是$\pm \infty$)时,$f,g$是等价的无穷小或无穷大,那么:
  
\begin{itemize}
\item $\lim\limits_{x \to x_0 }\frac{f(x)}{h(x)}=\lim\limits_{x \to x_0 }g(x)h(x)$;
\item $\lim\limits_{x \to x_0 }\frac{f(x)}{h(x)}=\lim\limits_{x \to x_0 }\frac{g(x)}{h(x)}$.
\end{itemize}
\end{theorem}
\begin{theorem}
 如果函数$f$与$g$在$x_0$处连续,那么$f \pm g$与$fg$都在$x_0$处连续,进一步,若$g(x_0) \neq 0$,则$\frac{f}{g}$也在$x_0$处连续.
\end{theorem}
\begin{theorem}
 设函数$g$在$t_0$处连续,记$g(t_0)$为$x_0$.如果函数$f$在$x_0$处连续,那么复合函数$f \circ g$在$t_0$处连续.
\end{theorem}
\begin{theorem}
 设$f$是在区间$I$上严格递增(减)的连续函数,那么$f^{-1}$是$f(I)$上的严格递增(减)的连续函数.
\end{theorem}
\end{document}
%%% Local Variables:
%%% mode: latex
%%% TeX-master: t
%%% End:
